\documentclass[]{article}   % list options between brackets
\usepackage{amsthm}              % list packages between braces
\usepackage{array}
\usepackage{amsmath}

% type user-defined commands here

\begin{document}

\title{\TeX\ and \LaTeX}   % type title between braces
\author{Robert E.Chen}         % type author(s) between braces
\date{October 27, 1995}    % type date between braces
\maketitle

\begin{abstract}
  We present Markov processes.
\end{abstract}

\section{Intro}     % section 1.1
The Charles River has been a landmark of Boston for several decades. Every fourth of July, fireworks are set off from a barge on the river while the Boston Pops Orchestra performs for half a million spectators.  Professional rowers as well as the MIT crew team train on the river. The river also serves as a swimming area for daring MIT students.  As well, the river is a playground for MIT chemistry students who want to understand the chemical contents of the river. 

It is important to monitor the presence of oxygen in the river, because the dissovled oxygen in the river provides nutrients to support certain aquatic life.  A satisfactory dissolved oxygen (DO) level that would be about 5.0 mg/L. Most fish are able to survive at this level. Some species such as trout and small mouth bass require levels of up to 6.5 mg/L; other species such as catfish and carp only require DO levels of 2.0-3.0 mg/L. If the level falls to 3.0-4.0 mg/L, the environment of the river could be harmful to most fish. A dissolved oxygen level of 2.0 mg/L would cause most fish to die. 

EEQUATION

	Another substance that should be analyzed is inorganic orthophosphate. Orthophosphate provides nutrients for eutrophication, which can be harmful to fish. Soil erosion as well as pollution from grass can lead to increased levels of orthophosphate. 

EQUATION 

	We analyzed the DO content of the river water using titration with a colorimetric titration. We also analyzed the orthophosphate content of the river water using a modified Molybdate Blue method, originally proposed by Strickland and Parsons for Seawater. [CITE]

\section{Procedures and Observations}
\subsection{Quantification of Orthophosphate}

\emph{Preparation of Charles River water Sample}\\
River from the Charles river was collected in 300mL BOD bottles that were previously rinsed with 10\% dilute HCl solution and dried. After the water samples were collected they were stored in a refigerator while being allowed to settle. After the water was allowed to settle, 10.0 mL of the water was pipetted into three separate small bottles.

\emph{Preparation of Color Developing Solutions (by TA)} \\
The TAs prepared the color developing solutions for the colorimetric assay. The color developing solution was prepared by mixing 2.6M sulfuric acid with ammonium molybdate, potassium antimonyl-tartrate ($C_8H_4K_2O_12Sb_2 \times 3 H_2O$), ascorbic acid. 

\emph{Preparation of 10\% HCl solution (by TA)}\\

\emph{Prepartion of Primary Standard Solution (by TA)}\\


To prepare the \emph{phosphate working standard solution}, a biological 1.00mL adjustable pipette was used to transfer 1.0mL of the TA's Primary Standard solution ($1x10^{-3}M$) to a 100mL volumetric flask previously rinsed with 10\% HCl solution and Milli-Q water. Milli-Q water was added afterwards to fill the flask up to the 100mL mark.


\emph{Preparation of Diluted Phosphate Standards from Stock Working Solution}\\
11 50mL beakers were set up for the assay. Seven of the beakers were used for the diluted phosphate standards. For these seven beakers, phosphate working stock solution was mixed with Milli-Q water in different ratios (see table \ref{table:stds}).

\begin{table}[h]
    \begin{tabular}{ | p{3cm} | p{3cm} | p{3cm} |}
    \hline
    Volume of $KH_2PO_4$ stock  & Volume of Milli-Q $H_2O$ to Add  & Final $PO_4^{3-}$ concentration \\ \hline
    0.00 mL & 10.00 mL & A: 0.00 $\mu M$ \\ \hline
    0.50 mL & 9.50 mL & B: 0.50 $\mu M$ \\ \hline
    1.00 mL & 9.00 mL & C: 1.00 $\mu M$ \\ \hline
    2.00 mL & 8.00 mL & D: 2.00 $\mu M$ \\ \hline
    4.00 mL & 6.00 mL & E: 4.00 $\mu M$ \\ \hline
    6.00 mL & 4.00 mL & F: 6.00 $\mu M$ \\ \hline
    8.00 mL & 2.00 mL & G: 8.00 $\mu M$ \\ \hline
    \end{tabular}
    \caption{Table of relative amounts of $KH_2PO_4$ and Milli-Q water in each standard solution.}
    \label{table:stds}
\end{table}





\begin{thebibliography}{9}
  % type bibliography here
\end{thebibliography}

\end{document}
